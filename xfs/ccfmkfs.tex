\documentclass[]{article}
\usepackage{titlesec}

\titleformat*{\section}{\LARGE\bfseries}
\titleformat*{\subsection}{\Large\bfseries}
\titleformat*{\subsubsection}{\large\bfseries}
\titleformat*{\paragraph}{\large\bfseries}
\titleformat*{\subparagraph}{\large\bfseries}

%opening
\title{%
\vspace{-30mm}\Huge
\textbf{mkfs.xfs code comment}\vspace{9mm}}
\author{root}
\begin{document}
\maketitle
\newpage
\section{Preface}
This is a code comment documentation on xfsprog-dev commit 3caeb69c19df7e955cd70f85ba361defc249dc11. \linebreak
If you have some suggestion, please email hmsjwzb@qq.com. Thank you!
\section{Codecomment}
\newcommand{\comment}[2]{\noindent \boldmath{#1:%
\parbox[t]{120mm}{\sl #2}}}

\newcommand{\summarize}[1]{\begin{center}
		\textbf{#1}
	\end{center}}
\comment{3733-3809}{Variable declaration and initialization}

\comment{3811-3861}{Get mkfs configuration from command line or configuration file}

\comment{3863-3907}{do some basic check on user argument}

\comment{3915-3918}{do some calculation on some}

\comment{3925-3926}{start\_superblock\_setup set up some part of super block(xfs\_sb)\newline initialise\_mount init some part of xfs\_mount}\\

\comment{3928-3934}{{\boldmath caculate\_log\_size} set up log location\newline {\boldmath finish\_superblock\_size} set up some members in super block}

\comment{3925-3953}{print out the geometry of the fs.}

\comment{3958}{allocate buffer cache for each device.}

\comment{3965}{zero out the device, It can protect xfs from some obsolete data.}

\comment{3966}{Compute and fill members in {xfs\_mount\_t}}

\comment{3976-3977}{Init static metadata for every AG}

\comment{3982-3983}{Init the freespace freelists for each AG, the main work is done in xfs\_alloc\_fix\_freelist}

\comment{3982-3983}{Initialise the freespace freelist (i.e. AGFLs) in each AG.}

\comment{3988}{The last thing allocate root inode for the filesystem}

\section{initialise\_ag\_headers}

\comment{3325}{Get ag management data structure for current AG}

\comment{3345}{Get cache buffer for Superblock}

\comment{3348}{Set the cache buffer operation for xfs\_sb\_buf\_ops}

\comment{3349}{Clear cache buffer to zero}

\comment{3350}{trans the xfs\_sb to the superblock disk cache buffer}

\comment{3356-3360}{Get a cache buffer for xfs\_agf\_t which manage the free space of the disk}

\comment{3361-3393}{Init element of the xfs\_agf\_t}

\comment{3396-3401}{Special treatment for log AG}

\comment{3409-3424}{Get cache buffer for AG freelist and init it to 0xff}

\comment{3428-3453}{Get and init cache buffer for agi of the AG which manage the inode of the AG}

\comment{3458-3505}{Get a cache buffer and do some init for BNO btree root block, It manage the free space of the AG by block number}

\comment{3507-3547}{Get a cache buffer and do some init for CNT btree root block, It manage the free space of the AG by extent length}

\comment{3551-3562}{If the filesystem is configured with refcount btree, then init it}

\comment{3567-3574}{Get a cache buffer for inode btree and init it}

\comment{3579-3588}{Get a free inode btree and init it if the filesystem is configured with free inode btree}

\comment{3591-3664}{If the filesystem is configured with RMAP btree, init it}

\section{xfs\_alloc\_fix\_freelist}
\comment{2237-2246}{If the AG isn't init, then init it}

\comment{2253-2257}{You can read the comment up there.}

\comment{2259}{Calculate the minimum blocks needed by various Btree.}

\comment{2260}{Check if we should go on}

\comment{2288-2378}{Make the free list shorter if it is too long! Make it longer if it is too short!}

\section{parseproto}
\comment{343-347}{We passing some arguments into parseproto, i.e. pp is default d--755 0 0 }
\comment{378-446}{Get some init attribute for root directory for allocation}
\comment{548}{Allocate xfs\_trans for the transaction}
\comment{549-550}{It allocate the inode for the root directory. The libxfs\_ialloc do the actual work}

\section{libxfs\_ialloc}
\summarize{SUMMARIZE: Allocate an inode on disk and return a copy of its in-core version.}
\comment{254-255}{Call the space management code to pick the on-disk inode}
\comment{1746-1762}{We do just like did in xfs\_ialloc\_ag\_select did}
\comment{1768-1776}{Read the AGI buffer in. If it contains free inode, just allocate from it.But it is not likely in the first time.}
\comment{1781}{Allocate new inodes from this allocation group.}


\section{xfs\_dialloc}
\comment{1704-1712}{If the AGI buffer passed is not NULL. We can grap an inode from it.}

\comment{1718}{Select an allocation group for this allocation.}

\comment{1724-1737}{If we hit the top of space allocation for inode, noroom set to 1 and okalloc set to 0}

\subsection{allocate through allocation group}
\comment{1745-1818}{allocate through allocation groups, xfs\_ialloc\_read\_agi will do the real work}
\comment{1746-1778}{Do some basic check and init work}
\comment{1781}{Do the actual allocate work}
\comment{1782-1791}{Check the allocate result}
\comment{1792-1806}{If the allocation is success, we return}

\section{xfs\_ialloc\_ag\_select}
\comment{952}{files, directories, symbolic link will need space}
\comment{955-956}{It is iterator that iterate all allocation group. In my PC, pagno will go through 0 to 4 and start form 0 again.}
\comment{974-1041}{The loop will examine the allocation group one by one}
\subsubsection{loop(974-1041)}
\comment{975}{Get the xfs\_perpag\_t for this allocation group}
\comment{975-979}{Is this allocation group ok for inode allocation?}
\comment{987-990}{Is this space contain free inode? If so, we just return this allocation group}
\comment{991-996}{If the allocation group describe buf is not in memory, We just read it in.}
\comment{998-1025}{If the allocation group has enough space for allocation, we return it.}
\comment{1027-1041}{If not, we start with next allocation group.}

\section{xfs\_ialloc\_ag\_alloc}
\summarize{SUMMARIZE:Allocate new inodes in the allocation group specified by agbp}
\comment{621-638}{Do some init operation.}
\comment{662-665}{Get some basic information from AGI}
\comment{669-707}{If the AGI is valid, we will go to this branch}
\comment{710-740}{Set the parameter for this allocation}
\comment{741-742}{Do this allocation}
\comment{814-815}{Init the inodes we just allocate}
\comment{872-884}{Insert the chunk of inodes we just allocate in b+tree}
\comment{890-896}{update the information in AGI}
\comment{901-909}{do some log operation and return}

\section{xfs\_ialloc\_inode\_init}
\comment{283-298}{do some init calculation}
\comment{319-332}{If we have crc feature, the version is 3 and we create log for it}
\comment{338-399}{Init the inode we just allocated one by one}
\end{document}
