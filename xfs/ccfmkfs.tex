\documentclass[]{article}

%opening
\title{%
\vspace{-30mm}\Huge
\textbf{mkfs.xfs code comment}\vspace{9mm}}
\author{root}
\begin{document}
\maketitle
\newpage
\section{Preface}
This is a code comment documentation on xfsprog-dev commit 3caeb69c19df7e955cd70f85ba361defc249dc11. \linebreak
If you have some suggestion, please email hmsjwzb@qq.com. Thank you!
\section{Codecomment}
\newcommand{\comment}[2]{\noindent \boldmath{#1:%
\parbox[t]{120mm}{\sl #2}}}

\comment{3733-3809}{Variable declaration and initialization}

\comment{3811-3861}{Get mkfs configuration from command line or configuration file}

\comment{3863-3907}{do some basic check on user argument}

\comment{3915-3918}{do some calculation on some}

\comment{3925-3926}{start\_superblock\_setup set up some part of super block(xfs\_sb)\newline initialise\_mount init some part of xfs\_mount}\\

\comment{3928-3934}{{\boldmath caculate\_log\_size} set up log location\newline {\boldmath finish\_superblock\_size} set up some members in super block}

\comment{3925-3953}{print out the geometry of the fs.}

\comment{3958}{allocate buffer cache for each device.}

\comment{3965}{zero out the device, It can protect xfs from some obsolete data.}

\comment{3966}{Compute and fill members in {xfs\_mount\_t}}

\comment{3976-3977}{Init static metadata for every AG}

\comment{3982-3983}{Init the freespace freelists for each AG, the main work is done in xfs\_alloc\_fix\_freelist}

\section{initialise\_ag\_headers}

\comment{3325}{Get ag management data structure for current AG}

\comment{3345}{Get cache buffer for Superblock}

\comment{3348}{Set the cache buffer operation for xfs\_sb\_buf\_ops}

\comment{3349}{Clear cache buffer to zero}

\comment{3350}{trans the xfs\_sb to the superblock disk cache buffer}

\comment{3356-3360}{Get a cache buffer for xfs\_agf\_t which manage the free space of the disk}

\comment{3361-3393}{Init element of the xfs\_agf\_t}

\comment{3396-3401}{Special treatment for log AG}

\comment{3409-3424}{Get cache buffer for AG freelist and init it to 0xff}

\comment{3428-3453}{Get and init cache buffer for agi of the AG which manage the inode of the AG}

\comment{3458-3505}{Get a cache buffer and do some init for BNO btree root block, It manage the free space of the AG by block number}

\comment{3507-3547}{Get a cache buffer and do some init for CNT btree root block, It manage the free space of the AG by extent length}

\comment{3551-3562}{If the filesystem is configured with refcount btree, then init it}

\comment{3567-3574}{Get a cache buffer for inode btree and init it}

\comment{3579-3588}{Get a free inode btree and init it if the filesystem is configured with free inode btree}

\comment{3591-3664}{If the filesystem is configured with RMAP btree, init it}

\section{xfs\_alloc\_fix\_freelist}
\comment{2237-2246}{If the AG isn't init, then init it}

\comment{2253-2257}{You can read the comment up there.}

\comment{2259}{Calculate the minimum blocks needed by various Btree.}

\comment{2260}{Check if we should go on}
\end{document}
